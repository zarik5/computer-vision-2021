\documentclass[11pt,a4paper]{article}
\usepackage[utf8]{inputenc}
\usepackage[british]{babel}
\usepackage{graphicx}
\usepackage{multicol, latexsym, amsmath, amssymb}
\usepackage{blindtext}
\usepackage{subcaption}
\usepackage{caption}
\usepackage{tabu}
\usepackage{booktabs}% for better rules in the table
\usepackage{anysize} % Soporte para el comando \marginsize
\usepackage{hyperref}
\setlength\parindent{0pt}
\usepackage[margin=1in]{geometry}

\newcommand{\cc}{\fontfamily{txtt}\selectfont}

\begin{document}
\begin{center}
\begin{tabular}{c}
\LARGE{Computer Vision LAB 3 Report}\\
\large{Mattia Giacomello}\\
I.D. 1210988
\end{tabular}
\end{center}

\section{Goal}
The objective of this homework is show the difference in histogram equalization applied in different color space and the implementation of three filters seen during class through a system of classes and their effect varying the parameters.

\section{Procedure}
We perform the histogram equalization in two different ways for the two color spaces considered, BGR and HSV, since the channels have different meaning for each color space.
For the first one, the histogram equalization is computed on each channel separately while for the second one the histograms equalization is computed on one channel at time while the other two are maintained without changes.
At first sight is clearly visible the RGB equalization changes the color of pixels greatly as in the figure (\ref{fig:BGREqual}).
This makes the images with warm tint more cold, since it push the histogram of blue channel that is condensed in the low values more high, making the opposite in the red channel histogram.
A similar phenomena is observable in an image with cold tint, that will be made warmer.\newline
Regarding the HSV color space, if the Hue channel is equalized we can observe that the images are still a little understandable but the color are all wrong since we are flattering the histogram of the colors channel.
In the Saturation channel in the first image we can observe that there is a improvement but in the other one, as in the hue case, the color are changed too much.
Considering the Value channel, that can be seen as the brightness of the pixels, we can observe a better enhancing of the quality of images \ref{fig:v}, since the overexposed image is darkened without much alteration of the colors.\newline
At this point the V equalized image, the best one till now, is filtered through three filters: Median, Gaussian and Bilateral filter.
To implement these filters are created four class, one for each filter and a generic {\cc Filter} class from which the other are derived.
This implementation makes managing the various filters simple.
The effects of the Median filter is noticeable in noisy images, since it removes pixels with high sudden changes.
We can observe that figure (\ref{fig:mednoise}) shows how the noise is less noticeable loosing a bit of quality.
Increasing the size the image became more blured, removing small details, till the elements became blobs of quasi constant color and difficult to distinguish.\newline
The Gaussian filter effect is more obvious: increasing the value of sigma and the window size the output image results more blured and the fine details are unrecognizable.
This helps to remove some noise and smooth squared colored areas obtained from the equalization.
If $\sigma$ is fixed and the size is low the effect is similar to a mean filter, since the values of kernel are quasi constant, while if the window exceed the 6$\sigma$ rule there are unnoticeable changes, since the kernel in the exceeding area is almost 0.\newline
For the Bilateral filter, to simplify the test, the size of filter is fixed to $6\sigma_{S}$. This is computationally heavy for large $\sigma_s$ values and for real time filtering will be better to set it to a fixed value, but for understanding better the filter effects is more interesting this setup.
The effect of this filter is lighter than the others and only for high parameter's values it's sensible its effect.
In particular it can remove small detail, for example the noise, maintaining the structure of most elements.
If we use a too small $\sigma_c$, even using a large $\sigma_s$, it is not so useful since the colors are still off in some pixels.
Choosing a good tradeoff is possible to observe that the example image is less grainy, maintaining recognizable the subject.
If the values are too high the image assume a cartoonish look as can be seen in figure (\ref{fig:cartoon}).

\begin{figure}[htbp]
\centering
\begin{minipage}{0.7\textwidth}
\begin{minipage}{0.4\textwidth}
  \centering
\includegraphics[width=0.98\textwidth]{Pictures/lena.jpg}
\end{minipage}%
\begin{minipage}{0.6\textwidth}
  \centering
\includegraphics[width=0.98\textwidth]{Pictures/overexposed.jpg}
\end{minipage}
\subcaption{Original Images}
\end{minipage}\\

\begin{minipage}{0.7\textwidth}
\begin{minipage}{0.4\textwidth}
  \centering
\includegraphics[width=0.98\textwidth]{Pictures/Lena_BGR_Equalized.jpg}
\end{minipage}%
\begin{minipage}{0.6\textwidth}
  \centering
\includegraphics[width=0.98\textwidth]{Pictures/Overexposed_BGR_Equalized.jpg}
\end{minipage}
\subcaption{BGR Equalized}\label{fig:BGREqual}
\end{minipage}\\

\begin{minipage}{0.7\textwidth}
\begin{minipage}{0.4\textwidth}
  \centering
\includegraphics[width=0.98\textwidth]{Pictures/HL.jpg}
\end{minipage}%
\begin{minipage}{0.6\textwidth}
  \centering
\includegraphics[width=0.98\textwidth]{Pictures/HO.jpg}
\end{minipage}
\subcaption{H Equalized}
\end{minipage}\\

\begin{minipage}{0.7\textwidth}
\begin{minipage}{0.4\textwidth}
  \centering
\includegraphics[width=0.98\textwidth]{Pictures/SL.jpg}
\end{minipage}%
\begin{minipage}{0.6\textwidth}
  \centering
\includegraphics[width=0.98\textwidth]{Pictures/SO.jpg}
\end{minipage}
\subcaption{S Equalized}
\end{minipage}\\

\begin{minipage}{0.7\textwidth}
  \begin{minipage}{0.4\textwidth}
    \centering
  \includegraphics[width=0.98\textwidth]{Pictures/Lena_V_Equalized.jpg}
  \end{minipage}%
  \begin{minipage}{0.6\textwidth}
    \centering
  \includegraphics[width=0.98\textwidth]{Pictures/Overexposed_V_Equalized.jpg}
  \end{minipage}
  \subcaption{V Equalized}\label{fig:v}
  \end{minipage}

\caption{Example of histograms equalization on different images} 
\end{figure}


\begin{figure}[htbp]
\centering
\begin{minipage}{0.5\textwidth}
  \centering
\includegraphics[width=0.98\textwidth]{Pictures/Meddenoise.jpg}
\subcaption{Size=5, denoise effcet}\label{fig:mednoise}
\end{minipage}%
\begin{minipage}{0.5\textwidth}
  \centering
\includegraphics[width=0.98\textwidth]{Pictures/MedHigh.jpg}
\subcaption{Size=15, high value}
\end{minipage}\\
\begin{center}
\begin{minipage}{0.5\textwidth}
  \centering
\includegraphics[width=0.98\textwidth]{Pictures/Med.jpg}
\subcaption{Size=50, blob effect}
  \end{minipage}
\end{center}
\caption{Example of Median filter with various size}
\end{figure}

\begin{figure}[htbp]
  \centering
  \begin{minipage}{0.5\textwidth}
    \centering
  \includegraphics[width=0.98\textwidth]{Pictures/GausLowSize.jpg}
  \subcaption{Size=11, low value}
  \end{minipage}%
  \begin{minipage}{0.5\textwidth}
    \centering
  \includegraphics[width=0.98\textwidth]{Pictures/GausGoodSize.jpg}
  \subcaption{Size=31$\approx$6$\sigma$, tipical choice}
  \end{minipage}\\
  \begin{center}
  \begin{minipage}{0.5\textwidth}
    \centering
  \includegraphics[width=0.98\textwidth]{Pictures/GausLargeSize.jpg}
  \subcaption{Size=61, large value}
    \end{minipage}
  \end{center}
  \caption{Example of Gaussian filter with $\sigma=5$ and changing size}
  \end{figure}

  \begin{figure}[htbp]
    \centering
    \begin{minipage}{0.5\textwidth}
      \centering
    \includegraphics[width=0.98\textwidth]{Pictures/Bil50_10.jpg}
    \subcaption{$\sigma_c$=10, $\sigma_s$=50}
    \end{minipage}%
    \begin{minipage}{0.5\textwidth}
      \centering
    \includegraphics[width=0.98\textwidth]{Pictures/Bil10_50.jpg}
    \subcaption{$\sigma_c$=50, $\sigma_s$=10}
    \end{minipage}\\

\begin{minipage}{0.5\textwidth}
  \centering
\includegraphics[width=0.98\textwidth]{Pictures/Bil3_150.jpg}
\subcaption{$\sigma_c$=150, $\sigma_s$=3}
\end{minipage}%
    \begin{minipage}{0.5\textwidth}
      \centering
    \includegraphics[width=0.98\textwidth]{Pictures/BilCartoon.jpg}
    \subcaption{$\sigma_c$=200, $\sigma_s$=5}\label{fig:cartoon}
      \end{minipage}
    \caption{Example with Bilater filter with various parameters}
    \end{figure} 

\end{document}
